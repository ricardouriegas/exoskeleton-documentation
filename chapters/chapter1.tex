\chapter{Planteamiento del problema}
\section{Introducción}
El trabajo manual implica esfuerzos físicos repetitivos y el transporte de herramientas que, si no se realiza de forma adecuada, puede causar lesiones graves, especialmente en la espalda. La ergonomía y la tecnología han avanzado para abordar estos desafíos, y el desarrollo de exoesqueletos surge como una solución innovadora. Este proyecto busca combinar principios de mecánica y tecnología de asistencia para crear un dispositivo que aligere la carga física y mejore las condiciones laborales.

\subsection{Antecedentes} % no mas de 1 cuartilla
En las últimas décadas, los exoesqueletos han evolucionado de simples soportes mecánicos a dispositivos complejos que integran tecnología avanzada para asistencia y rehabilitación. Estos dispositivos, utilizados principalmente en el ámbito industrial y médico, han demostrado ser efectivos para reducir lesiones y aumentar la productividad. Sin embargo, la mayoría de los exoesqueletos disponibles están diseñados para operaciones industriales pesadas o rehabilitación física, dejando un vacío en el mercado para dispositivos ligeros, accesibles y orientados a tareas domésticas o laborales de baja intensidad.

El desarrollo de un exoesqueleto inspirado en sistemas de grúas, acoplado a una faja de corrección postural, podría ofrecer una solución práctica para trabajadores que manejan herramientas ligeras. Este enfoque no solo aborda las limitaciones actuales de diseño, sino que también aprovecha tecnologías de detección e inteligencia artificial para mejorar la interacción humano-máquina y garantizar la seguridad del usuario.


\subsection{Planteamiento del problema}
Se busca desarrollar un exoesqueleto que permita a un trabajador doméstico o profesional cargar herramientas no mayores a 10 kg con facilidad. Este dispositivo debe ser capaz de tomar la herramienta directamente del usuario, alejándola del cuerpo y manteniendo la postura correcta para evitar lesiones de espalda. Actualmente, las herramientas diseñadas para tareas manuales no incluyen soporte físico para cargas ligeras ni corrección postural integrada, lo que genera desgaste físico y riesgo de lesiones a largo plazo.

\subsection{Solución del Planteamiento del Problema} % 1/2 cuartilla
% TODO: Mencionar lo que lleva el cap 1 y los demás capítulos

\section{Objetivos}
\subsection{General}
Diseñar y desarrollar un exoesqueleto inspirado en una grúa, acoplado a una faja de corrección postural, para facilitar el manejo de herramientas ligeras (hasta 10 kg) mientras se promueve una postura adecuada del usuario.

\subsection{Específicos}
\begin{itemize}
    \item Diseñar un prototipo ergonómico que integre mecánica estructural y corrección postural.
    \item Implementar una inteligencia artificial capaz de detectar herramientas y ejecutar maniobras para cargar y posicionarlas.
    \item Validar la capacidad del exoesqueleto para reducir el esfuerzo físico del usuario mediante pruebas experimentales.
    \item Analizar los riesgos ergonómicos asociados al diseño y minimizar las probabilidades de lesiones.
\end{itemize}

\section{Hipótesis Descriptiva}
Un exoesqueleto ligero inspirado en una grúa y diseñado con tecnología de corrección postural reducirá significativamente el esfuerzo físico y el riesgo de lesiones en la espalda para trabajadores que manipulan herramientas de hasta 10 kg.


\section{Preguntas de Investigación}
\begin{itemize}
    \item ¿Qué diseño estructural es más adecuado para un exoesqueleto inspirado en grúas y enfocado en cargas ligeras?
    \item ¿Cómo integrar un sistema de inteligencia artificial para detección y manipulación de herramientas ligeras?
    \item ¿Qué materiales ofrecen una mayor relación entre resistencia y peso para un dispositivo ergonómico?
    \item ¿Qué impactos físicos y ergonómicos se esperan al usar el dispositivo?
\end{itemize}


\section{Justificación}
\subsection{Antecedentes}
Las lesiones de espalda son una de las principales causas de incapacidad laboral y representan un costo significativo para los sistemas de salud y las empresas. La necesidad de herramientas ergonómicas para tareas manuales no solo busca prevenir lesiones, sino también aumentar la productividad y comodidad del usuario. Este proyecto combina tecnologías modernas con un enfoque práctico y accesible para desarrollar un dispositivo adaptado a tareas cotidianas, abriendo nuevas posibilidades en la integración de exoesqueletos en actividades domésticas y laborales ligeras.


\section{Alcances y Limitaciones}
\subsection{Alcances}
\begin{itemize}
    \item Desarrollo de un prototipo funcional que incluya sistemas de detección, soporte mecánico y corrección postural.
    \item Evaluación del desempeño del exoesqueleto en un entorno controlado con usuarios representativos.
\end{itemize}

\subsection{Limitaciones}
\begin{itemize}
    \item El diseño se limitará a cargas no mayores a 10 kg.
    \item El proyecto no contempla la fabricación a escala industrial del dispositivo, enfocándose únicamente en un prototipo funcional.
\end{itemize}


\section{Factibilidad}
% TODO: que tan viable es el proyecto