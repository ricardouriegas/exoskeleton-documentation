\chapter{Marco teórico y estado del arte}

\section{Marco teórico}
El marco teórico proporciona la base conceptual que respalda el desarrollo de este proyecto. La idea del exoesqueleto inspirado en una grúa requiere integrar conceptos de mecánica, ergonomía, biomecánica e inteligencia artificial, los cuales se detallan a continuación:

\subsection{Mecánica estructural}
La mecánica estructural estudia el comportamiento de estructuras sometidas a diferentes fuerzas. En este caso, se aplican principios de sistemas de grúas ligeras para garantizar que el exoesqueleto soporte las cargas propuestas sin comprometer su peso ni movilidad \cite{smith2019structuraldesign, lee2020lightweightsystems}.

\subsection{Ergonomía y corrección postural}
La ergonomía analiza las interacciones entre los usuarios y su entorno, orientándose a reducir lesiones y mejorar el rendimiento. Este proyecto incorpora una faja ergonómica para promover una postura adecuada, mitigando el esfuerzo físico en la región lumbar \cite{brown2022ergonomics, liu2023posturalstudy}.

\subsection{Inteligencia artificial y visión por computadora}
El uso de inteligencia artificial permite detectar herramientas en tiempo real mediante visión por computadora. Esto asegura que el exoesqueleto identifique, tome y manipule las herramientas sin intervención directa del usuario \cite{redmon2018yolo, patel2021tooldetection}.

\subsection{Materiales y diseño ligero}
El diseño del dispositivo requiere materiales con una alta relación resistencia-peso, como aleaciones ligeras y polímeros reforzados, para garantizar que el exoesqueleto sea cómodo y funcional \cite{choi2020carbonfiber, park2019lightweight, nguyen2020visionrobotics}.

\section{Estado del arte}
El estado del arte de este proyecto abarca los avances recientes en áreas relacionadas con el diseño de exoesqueletos, manejo ergonómico de cargas, tecnologías de visión por computadora, y materiales ligeros. Este análisis se enfoca en investigaciones realizadas en los últimos cinco años, ampliando el rango temporal si es necesario.

\subsection{Exoesqueletos industriales y aplicaciones laborales}
En los últimos años, los exoesqueletos han evolucionado significativamente para abordar necesidades específicas en aplicaciones laborales. Los desarrollos incluyen tanto exoesqueletos activos como pasivos:
\begin{itemize}
    \item \textbf{Exoesqueletos activos}: Estos utilizan actuadores motorizados para proporcionar asistencia. Ejemplos recientes incluyen el \textit{EksoVest} \cite{garcia2020integratedai} y el \textit{Comau MATE}, diseñados para tareas repetitivas y pesadas \cite{ramirez2021specifictasks}.
    \item \textbf{Exoesqueletos pasivos}: Sistemas como el \textit{HeroWear Apex} emplean estructuras mecánicas que no requieren energía externa, mejorando la accesibilidad por su costo reducido \cite{smith2021ergonomic, gomez2022hybridsystems}.
\end{itemize}

Estudios recientes como los de \cite{kim2018overview, johnson2018biomechanics} destacan la adopción de exoesqueletos en la industria de la construcción para mitigar el esfuerzo físico en trabajadores, aunque existe un vacío en dispositivos diseñados para manipular herramientas ligeras.

\subsection{Corrección postural y ergonomía}
La corrección postural es un componente clave para prevenir lesiones. Investigaciones recientes han explorado tecnologías portátiles para este propósito \cite{liu2023posturalstudy, choi2020carbonfiber}:
\begin{itemize}
    \item \textbf{Fajas ergonómicas inteligentes}: Incorporan sensores para monitorear la postura y emitir alertas si se detectan posiciones incorrectas \cite{brown2022ergonomics}.
    \item \textbf{Evaluación biomecánica}: Demuestran que dispositivos ergonómicos bien diseñados pueden reducir significativamente las tensiones en la columna vertebral \cite{johnson2018biomechanics}.
\end{itemize}

\subsection{Materiales ligeros para exoesqueletos}
El peso del exoesqueleto es un factor crítico para su aceptación. Los avances en materiales han permitido la creación de dispositivos más ligeros y cómodos \cite{park2019lightweight, nguyen2020visionrobotics}:
\begin{itemize}
    \item \textbf{Polímeros avanzados}: Como los compuestos de fibra de carbono, ofrecen alta resistencia con bajo peso.
    \item \textbf{Aleaciones de aluminio y titanio}: Utilizados para reducir el peso estructural, como en el \textit{HAL} de Cyberdyne \cite{choi2020carbonfiber}.
    \item \textbf{Textiles inteligentes}: Emplean tejidos con sensores integrados para facilitar la detección de movimientos \cite{nguyen2020visionrobotics}.
\end{itemize}

\subsection{Visión por computadora y detección de objetos}
La visión por computadora juega un papel fundamental en la automatización de sistemas portátiles. Avances recientes incluyen \cite{redmon2018yolo, patel2021tooldetection}:
\begin{itemize}
    \item \textbf{Detección en tiempo real}: Algoritmos como YOLO y SSD son ampliamente adoptados en aplicaciones industriales.
    \item \textbf{Reconocimiento de actividad}: Modelos de aprendizaje profundo para identificar tareas específicas realizadas por el usuario.
    \item \textbf{Sistemas híbridos}: Combinan visión por computadora con sensores hápticos para mejorar la precisión.
\end{itemize}

\subsection{Avances en inteligencia artificial aplicada a exoesqueletos}
La inteligencia artificial ha transformado la interacción usuario-exoesqueleto \cite{liu2021costanalysis, garcia2020integratedai}:
\begin{itemize}
    \item \textbf{Redes neuronales}: Modelos que anticipan movimientos del usuario para proporcionar soporte en tiempo real.
    \item \textbf{Control adaptable}: Ajustan el nivel de asistencia basado en las condiciones de carga y las necesidades específicas del usuario.
\end{itemize}

\section{Limitaciones y áreas de oportunidad}
A pesar de los avances, persisten desafíos como altos costos y diseños no inclusivos, así como la falta de integración de tecnologías ligeras y avanzadas. Los próximos desarrollos deben abordar estos desafíos para lograr una adopción más amplia.