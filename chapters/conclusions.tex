\chapter{Conclusiones}
\section{Conclusiones Alcanzadas}

El desarrollo del exoesqueleto inspirado en sistemas de grúas, acoplado a una faja de corrección postural, ha demostrado ser una solución efectiva para reducir el esfuerzo físico y prevenir lesiones en la espalda de trabajadores que manejan herramientas ligeras. A continuación, se detallan las conclusiones alcanzadas a lo largo del proyecto:

\subsection{Reducción del Esfuerzo Físico}
Mediante las pruebas experimentales realizadas, se observó una disminución promedio del 60\% en la carga percibida por los usuarios al utilizar el exoesqueleto. Este resultado confirma que el diseño mecánico y la integración de la inteligencia artificial logran mitigar significativamente el esfuerzo físico requerido para manejar herramientas de hasta 10 kg, alineándose con el objetivo general del proyecto.

\subsection{Eficiencia Económica}
El análisis comparativo de costos reveló que el prototipo desarrollado tiene un costo estimado de \$10,000 MXN, en contraste con los exoesqueletos comerciales que superan los \$50,000 MXN. La utilización de materiales accesibles y componentes estándar permitió alcanzar una solución económicamente viable, lo que amplía el potencial de acceso a esta tecnología en entornos laborales y domésticos.

\subsection{Mejora de la Postura}
La integración de la faja ergonómica con el sistema de soporte mecánico garantizó una postura adecuada durante el uso del exoesqueleto. Las mediciones realizadas demostraron que los usuarios mantuvieron una alineación correcta de la columna vertebral, reduciendo el riesgo de lesiones a largo plazo y promoviendo hábitos posturales saludables.

\subsection{Funcionalidad de la Inteligencia Artificial}
La implementación de algoritmos de inteligencia artificial para controlar la retracción y extracción de la cuerda que sostiene la herramienta ha sido exitosa. El sistema demostró una respuesta rápida y precisa a las acciones del usuario, facilitando el manejo fluido de las herramientas y mejorando la interacción humano-máquina.

\subsection{Viabilidad Técnica y Operativa}
El prototipo desarrollado es técnicamente viable, cumpliendo con los requisitos de diseño y funcionalidad establecidos. Además, la facilidad de uso y la modularidad del diseño permiten una adaptación rápida a diferentes tipos de herramientas y entornos de trabajo, aumentando su aplicabilidad y usabilidad.

\subsection{Validación de la Hipótesis}
Los resultados obtenidos validan la hipótesis planteada inicialmente: el exoesqueleto diseñado no solo reduce el esfuerzo físico del usuario, sino que también es una solución económicamente accesible en comparación con las opciones comerciales existentes. Esta validación respalda la efectividad y el impacto positivo del proyecto en la mejora de las condiciones laborales.

\subsection{Cumplimiento de los Objetivos Específicos}
Todos los objetivos específicos propuestos fueron alcanzados de manera satisfactoria:
\begin{itemize}
    \item \textbf{Diseño Ergonómico:} Se desarrolló un prototipo que integra principios de ergonomía, asegurando confort y funcionalidad para el usuario.
    \item \textbf{Implementación de Inteligencia Artificial:} Se incorporaron algoritmos de IA que controlan de manera eficiente el movimiento de la herramienta.
    \item \textbf{Validación Experimental:} Las pruebas demostraron una reducción significativa en el esfuerzo físico, cumpliendo con los objetivos de validación presentados.
    \item \textbf{Análisis de Riesgos Ergonómicos:} Se identificaron y minimizaron los riesgos ergonómicos, garantizando la seguridad y salud de los usuarios.
\end{itemize}

\subsection{Impacto en el Entorno Laboral}
El exoesqueleto desarrolló aporta beneficios tangibles al entorno laboral, tales como la disminución de lesiones musculoesqueléticas, aumento de la productividad y mejora en la calidad de vida de los trabajadores. Su implementación puede transformar prácticas laborales, promoviendo un ambiente de trabajo más seguro y eficiente.

\subsection{Líneas de Trabajo Futuro}

El desarrollo del exoesqueleto inspirado en sistemas de grúas ha sentado las bases para futuras investigaciones y mejoras en diversas áreas. A continuación, se detallan las principales líneas de trabajo que podrían contribuir a la evolución y optimización de este proyecto:

\subsubsection{Optimización de la Autonomía de la Batería}
Una de las limitaciones identificadas durante el desarrollo fue la duración de la batería, la cual actualmente permite un funcionamiento continuo de aproximadamente 4 horas. Futuras investigaciones podrían enfocarse en la integración de baterías de mayor capacidad o en la implementación de sistemas de gestión de energía más eficientes. Además, explorar tecnologías de baterías de nueva generación, como las baterías de estado sólido, podría incrementar significativamente la autonomía del dispositivo.

\subsubsection{Integración de Sensores Avanzados}
Para mejorar la precisión y la adaptabilidad del exoesqueleto, se propone la incorporación de sensores más avanzados, tales como sensores de presión, giroscopios y acelerómetros de alta resolución. Estos sensores permitirían un monitoreo más detallado de los movimientos del usuario y de las condiciones ambientales, facilitando ajustes en tiempo real que optimicen el soporte proporcionado por el dispositivo.

\subsubsection{Desarrollo de Algoritmos de Inteligencia Artificial Más Sofisticados}
Aunque los algoritmos de inteligencia artificial implementados han demostrado ser efectivos en la retracción y extracción de la cuerda, existe un amplio margen para mejorar su sofisticación. La incorporación de técnicas de aprendizaje profundo y redes neuronales recurrentes podría permitir al exoesqueleto anticipar mejor las necesidades del usuario y adaptarse dinámicamente a diferentes tipos de herramientas y condiciones de trabajo.

\subsubsection{Mejora de la Ergonomía y Confort del Usuario}
El confort durante el uso prolongado del exoesqueleto es crucial para su adopción en entornos laborales. Investigaciones futuras podrían explorar materiales más ligeros y transpirables, así como diseños ajustables que se adapten a diferentes tipos de cuerpos y posturas. Además, la implementación de mecanismos de amortiguación y distribución de peso más eficientes podría reducir aún más la fatiga del usuario.

\subsubsection{Ampliación de las Capacidades Funcionales}
Actualmente, el exoesqueleto está diseñado para manejar herramientas de hasta 10 kg. Futuras versiones podrían ampliar este rango de capacidad, permitiendo el manejo de herramientas de diferentes tamaños y pesos. Asimismo, se podría explorar la posibilidad de incorporar funcionalidades adicionales, como sistemas de almacenamiento de herramientas integrados o mecanismos de asistencia para tareas específicas.

\subsubsection{Validación en Entornos Reales de Trabajo}
Para garantizar la efectividad y la aplicabilidad del exoesqueleto, es fundamental llevar a cabo pruebas exhaustivas en entornos laborales reales. Estas pruebas permitirían identificar posibles mejoras en el diseño y funcionalidad del dispositivo, así como evaluar su impacto en la productividad y el bienestar de los trabajadores. Además, la retroalimentación obtenida de los usuarios finales sería invaluable para realizar ajustes que aseguren una implementación exitosa.

\subsubsection{Estudio de Impacto a Largo Plazo en la Salud del Usuario}
Es importante evaluar los efectos a largo plazo del uso del exoesqueleto en la salud de los usuarios. Investigaciones futuras podrían enfocarse en estudios longitudinales que examinen cómo el dispositivo influye en la prevención de lesiones musculoesqueléticas y en la mejora de la postura a lo largo del tiempo. Estos estudios proporcionarían datos cruciales para validar los beneficios ergonómicos del exoesqueleto y para realizar ajustes que maximicen dichos beneficios.

\subsubsection{Exploración de Nuevas Aplicaciones y Mercados}
Finalmente, se sugiere explorar nuevas aplicaciones del exoesqueleto en diferentes industrias y sectores, más allá del ámbito doméstico y laboral inicialmente contemplado. Por ejemplo, su uso en el sector de la construcción, la agricultura o incluso en actividades recreativas podría abrir nuevas oportunidades de mercado. Adaptar el diseño para satisfacer las necesidades específicas de estas áreas contribuiría a la diversificación y sostenibilidad del proyecto.
