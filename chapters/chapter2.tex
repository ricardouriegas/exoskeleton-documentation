\chapter{Marco Teórico y Estado del Arte}

\section{Marco Teórico}

\subsection{Antecedentes}
Los exoesqueletos han evolucionado significativamente en los últimos años, con aplicaciones en diversas industrias \cite{Anderson2024}. Estudios previos han demostrado mejoras en la eficiencia y seguridad laboral mediante el uso de exoesqueletos \cite{Brown2025}.
\begin{itemize}
    \item \textbf{Industria:} Uso para tareas repetitivas y soporte físico en fábricas.
    \item \textbf{Medicina:} Rehabilitación de pacientes con movilidad reducida.
    \item \textbf{Militar:} Incremento en la fuerza y resistencia de los soldados.
\end{itemize}

Dispositivos como el \textit{EksoVest} y el \textit{ReWalk} han servido como pioneros. Estudios iniciales en grúas mecánicas establecieron bases biomecánicas utilizadas en exoesqueletos actuales, enfatizando la redistribución de peso y eficiencia estructural.

\subsection{Teoría}
La investigación de exoesqueletos se fundamenta en:
\begin{itemize}
    \item \textbf{Biomecánica:} Estudio de la interacción entre el cuerpo humano y sistemas mecánicos.
    \item \textbf{Teoría de Control:} Implementación de sistemas retroalimentados que ajustan el soporte en tiempo real.
    \item \textbf{Materiales Inteligentes:} Uso de materiales ligeros y con memoria de forma para mejorar funcionalidad.
    \item \textbf{Ergonomía:} Principios de adaptación al cuerpo humano para minimizar fatiga.
\end{itemize}

\subsection{Modelos}
Para diseñar y optimizar exoesqueletos, se utilizan modelos específicos:
\begin{itemize}
    \item \textbf{Modelo Cinemático de Grúas:} Simula puntos de apoyo y distribución de fuerzas.
    \item \textbf{Modelos Biomecánicos:} Sincronizan dinámicas corporales con el dispositivo.
    \item \textbf{Modelos Predictivos con IA:} Algoritmos como redes neuronales para anticipar movimientos y ajustar el soporte.
\end{itemize}

\subsection{Métodos}
El desarrollo de exoesqueletos implica:
\begin{itemize}
    \item \textbf{Análisis Experimental:} Evaluación en condiciones reales.
    \item \textbf{Simulaciones Computacionales:} Uso de herramientas como \textit{MATLAB} o \textit{ANSYS}.
    \item \textbf{Evaluaciones Ergonométricas:} Análisis de comodidad y aceptación del usuario mediante encuestas y sensores.
\end{itemize}

\subsection{Metodología}
El enfoque metodológico incluye:
\begin{enumerate}
    \item \textbf{Investigación Exploratoria:} Revisión de literatura sobre tecnologías y limitaciones.
    \item \textbf{Diseño y Prototipado:} Creación de modelos inspirados en grúas utilizando materiales ligeros.
    \item \textbf{Validación Experimental:} Estudios de caso en entornos industriales y médicos.
    \item \textbf{Análisis Cuantitativo:} Recolección y evaluación de datos, como reducción de esfuerzo y eficiencia energética.
\end{enumerate}

\section*{Estado del Arte}

\subsection*{Exoesqueletos Industriales}
En el ámbito industrial, los exoesqueletos se utilizan para mitigar el esfuerzo físico durante tareas repetitivas o pesadas. Estudios recientes destacan dispositivos como el \textbf{SuitX} y el \textbf{EksoVest}, diseñados para reducir el estrés en la columna y los miembros superiores. Estas soluciones han demostrado reducir significativamente el riesgo de lesiones musculoesqueléticas y aumentar la productividad en entornos de manufactura y construcción. Sin embargo, aún enfrentan desafíos relacionados con la adaptabilidad y la comodidad del usuario.

\subsection*{Inspiración en Sistemas de Grúas}
La inspiración en las grúas ha llevado al diseño de exoesqueletos que aprovechan la biomecánica para distribuir el peso de manera eficiente. Esto es particularmente útil en tareas donde se requiere una carga constante y sostenida. Prototipos recientes combinan mecanismos articulados ligeros con sensores inteligentes para optimizar el soporte físico y la movilidad, abordando los problemas de peso y rigidez presentes en modelos más antiguos.

\subsection*{Aplicaciones Médicas y Rehabilitación}
En el área de salud, los exoesqueletos han sido integrados para apoyar la rehabilitación física. Dispositivos como el \textbf{ReWalk} y el \textbf{Lokomat} han facilitado la recuperación en pacientes con movilidad reducida. Investigaciones actuales incluyen sensores avanzados para monitorear la interacción entre el cuerpo y el dispositivo, mejorando así la personalización del soporte terapéutico.

\subsection*{Inteligencia Artificial en Exoesqueletos}
La incorporación de inteligencia artificial (IA) ha revolucionado las capacidades de los exoesqueletos. Algoritmos de aprendizaje automático permiten ajustes en tiempo real basados en la actividad del usuario, mientras que tecnologías de visión computacional, como \textbf{YOLO} y \textbf{SSD}, habilitan el reconocimiento de tareas y objetos en tiempo real. Esto mejora la funcionalidad y aumenta la seguridad en el uso del dispositivo.

\subsection*{Limitaciones y Desafíos Actuales}
A pesar de los avances, persisten barreras importantes:
\begin{itemize}
    \item \textbf{Costo elevado:} Los dispositivos siguen siendo inaccesibles para muchas empresas y usuarios individuales.
    \item \textbf{Autonomía limitada:} La duración de la batería restringe el tiempo de uso continuo.
    \item \textbf{Diseño no inclusivo:} Falta de adaptabilidad a diversos tipos de cuerpo y condiciones de trabajo.
\end{itemize}

\subsection*{Innovaciones Recientes}
\begin{itemize}
    \item \textbf{Materiales ligeros y resistentes:} El uso de compuestos avanzados ha permitido la creación de dispositivos más livianos sin sacrificar la durabilidad.
    \item \textbf{Diseños modulares:} Facilitan la personalización y mantenimiento, aumentando la versatilidad en aplicaciones industriales y médicas.
    \item \textbf{Interfaces amigables:} La integración de controles intuitivos ha mejorado la experiencia del usuario y su adopción generalizada.
\end{itemize}
