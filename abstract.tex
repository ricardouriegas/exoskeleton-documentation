\begin{abstract}
    El presente proyecto aborda el diseño y desarrollo de un exoesqueleto ligero inspirado en sistemas de grúas \cite{Smith2020}, destinado a reducir el esfuerzo físico y prevenir lesiones en la espalda de trabajadores que manejan herramientas ligeras \cite{Johnson2021}. A lo largo del documento, se detalla el planteamiento del problema, los antecedentes y la justificación del proyecto, así como los objetivos generales y específicos.
    
    El marco teórico proporciona una visión integral de la evolución de los exoesqueletos en diversos campos, incluyendo la industria, la medicina y el ámbito militar. Se exploran teorías fundamentales como la biomecánica, la teoría de control y la ergonomía, además de modelos y métodos utilizados en el diseño y optimización de exoesqueletos.
    
    La metodología de la investigación se presenta mediante diagramas de flujo, proceso y bloque, que ilustran las etapas del desarrollo del proyecto, desde la identificación del problema hasta la validación del prototipo. Se describen las pruebas experimentales realizadas para evaluar la reducción del esfuerzo físico, la corrección postural y la viabilidad económica del dispositivo.
    
    Los resultados obtenidos confirman que el exoesqueleto desarrollado es efectivo en la reducción del esfuerzo físico, mejora la postura del usuario y es económicamente accesible en comparación con soluciones comerciales existentes. Las pruebas funcionales con usuarios reales validaron la facilidad de uso y los beneficios del dispositivo.
    
    Finalmente, se discuten las conclusiones alcanzadas y se proponen líneas de trabajo futuro para optimizar la autonomía de la batería, integrar sensores avanzados y explorar nuevas aplicaciones del exoesqueleto en diferentes industrias. El proyecto demuestra ser viable tanto en diseño como en implementación, abriendo nuevas posibilidades para la mejora de las condiciones laborales y la prevención de lesiones.
\end{abstract}