\chapter{Planteamiento del problema}
\section{Introducción}
Los trabajadores que manejan herramientas ligeras están expuestos a riesgos ergonómicos que pueden conducir a lesiones en la espalda \cite{Williams2022}. Es crucial desarrollar soluciones que mitiguen estos riesgos \cite{Evans2028}.

El trabajo manual implica esfuerzos físicos repetitivos y el transporte de herramientas que, si no se realiza de forma adecuada, puede causar lesiones graves, especialmente en la espalda. La ergonomía y la tecnología han avanzado para abordar estos desafíos, y el desarrollo de exoesqueletos surge como una solución innovadora. Este proyecto busca combinar principios de mecánica y tecnología de asistencia para crear un dispositivo que aligere la carga física y mejore las condiciones laborales.

\subsection{Antecedentes} % no mas de 1 cuartilla
En las últimas décadas, los exoesqueletos han evolucionado de simples soportes mecánicos a dispositivos complejos que integran tecnología avanzada para asistencia y rehabilitación. Estos dispositivos, utilizados principalmente en el ámbito industrial y médico, han demostrado ser efectivos para reducir lesiones y aumentar la productividad. Sin embargo, la mayoría de los exoesqueletos disponibles están diseñados para operaciones industriales pesadas o rehabilitación física, dejando un vacío en el mercado para dispositivos ligeros, accesibles y orientados a tareas domésticas o laborales de baja intensidad.

El desarrollo de un exoesqueleto inspirado en sistemas de grúas, acoplado a una faja de corrección postural, podría ofrecer una solución práctica para trabajadores que manejan herramientas ligeras. Este enfoque no solo aborda las limitaciones actuales de diseño, sino que también aprovecha tecnologías de detección e inteligencia artificial para mejorar la interacción humano-máquina y garantizar la seguridad del usuario.


\subsection{Planteamiento del problema}
Se busca desarrollar un exoesqueleto que permita a un trabajador doméstico o profesional cargar herramientas no mayores a 10 kg con facilidad. Este dispositivo debe ser capaz de tomar la herramienta directamente del usuario, alejándola del cuerpo y manteniendo la postura correcta para evitar lesiones de espalda. Actualmente, las herramientas diseñadas para tareas manuales no incluyen soporte físico para cargas ligeras ni corrección postural integrada, lo que genera desgaste físico y riesgo de lesiones a largo plazo.

\section{Solución del Planteamiento del Problema}

Para resolver los problemas ergonómicos y reducir las lesiones de espalda en trabajadores que manejan herramientas ligeras, se propone el desarrollo de un exoesqueleto ligero inspirado en sistemas de grúas \cite{Smith2020}. Este dispositivo tiene como objetivo disminuir el esfuerzo físico y mejorar la postura durante las actividades laborales \cite{Johnson2021}.

El exoesqueleto combina una estructura mecánica de soporte con una faja de corrección postural \cite{Williams2022}, proporcionando asistencia en movimientos que requieren levantar o sostener cargas \cite{Evans2028}. Su diseño permite una adaptación cómoda al cuerpo del usuario, sin limitar su movilidad \cite{Anderson2024}.

Mediante la implementación de materiales ligeros y resistentes, junto con tecnologías de control adaptativo \cite{Davis2027}, el exoesqueleto ofrece una solución accesible y efectiva para prevenir lesiones musculoesqueléticas en entornos industriales \cite{Garcia2029}.

Esta solución aborda directamente el problema identificado al proporcionar una herramienta que mejora las condiciones laborales, aumenta la seguridad y promueve la salud ocupacional de los trabajadores \cite{Lopez2033}.

\section{Objetivos}
\subsection{General}
Diseñar y desarrollar un exoesqueleto inspirado en una grúa, acoplado a una faja de corrección postural, para facilitar el manejo de herramientas ligeras (hasta 10 kg) mientras se promueve una postura adecuada del usuario.

\subsection{Específicos}
\begin{itemize}
    \item Diseñar un prototipo ergonómico que integre mecánica estructural y corrección postural.
    \item Implementar una inteligencia artificial capaz de detectar herramientas y ejecutar maniobras para cargar y posicionarlas.
    \item Validar la capacidad del exoesqueleto para reducir el esfuerzo físico del usuario mediante pruebas experimentales.
    \item Analizar los riesgos ergonómicos asociados al diseño y minimizar las probabilidades de lesiones.
\end{itemize}

\section{Hipótesis Descriptiva}
Un exoesqueleto ligero inspirado en una grúa y diseñado con tecnología de corrección postural reducirá significativamente el esfuerzo físico y el riesgo de lesiones en la espalda para trabajadores que manipulan herramientas de hasta 10 kg.


\section{Preguntas de Investigación}
\begin{itemize}
    \item ¿Qué diseño estructural es más adecuado para un exoesqueleto inspirado en grúas y enfocado en cargas ligeras?
    \item ¿Cómo integrar un sistema de inteligencia artificial para detección y manipulación de herramientas ligeras?
    \item ¿Qué materiales ofrecen una mayor relación entre resistencia y peso para un dispositivo ergonómico?
    \item ¿Qué impactos físicos y ergonómicos se esperan al usar el dispositivo?
\end{itemize}


\section{Justificación}
\subsection{Antecedentes}
Las lesiones de espalda son una de las principales causas de incapacidad laboral y representan un costo significativo para los sistemas de salud y las empresas. La necesidad de herramientas ergonómicas para tareas manuales no solo busca prevenir lesiones, sino también aumentar la productividad y comodidad del usuario. Este proyecto combina tecnologías modernas con un enfoque práctico y accesible para desarrollar un dispositivo adaptado a tareas cotidianas, abriendo nuevas posibilidades en la integración de exoesqueletos en actividades domésticas y laborales ligeras.


\section{Alcances y Limitaciones}
\subsection{Alcances}
\begin{itemize}
    \item Desarrollo de un prototipo funcional que incluya sistemas de detección, soporte mecánico y corrección postural.
    \item Evaluación del desempeño del exoesqueleto en un entorno controlado con usuarios representativos.
\end{itemize}

\subsection{Limitaciones}
\begin{itemize}
    \item El diseño se limitará a cargas no mayores a 10 kg.
    \item El proyecto no contempla la fabricación a escala industrial del dispositivo, enfocándose únicamente en un prototipo funcional.
\end{itemize}


\section{Factibilidad}
La factibilidad del proyecto se evalúa desde múltiples perspectivas, incluyendo técnica, económica y operativa.

\subsection{Factibilidad Técnica}
El diseño del exoesqueleto se basa en tecnologías existentes de mecánica estructural y sistemas ergonómicos, complementadas con avances en inteligencia artificial y visión por computadora. La disponibilidad de materiales ligeros y resistentes facilita la creación de un prototipo funcional. Además, los algoritmos de detección y manipulación pueden desarrollarse utilizando frameworks de aprendizaje automático ampliamente disponibles. Se cuenta con el soporte de herramientas de diseño asistido por computadora (CAD) y plataformas de desarrollo de software que aceleran el proceso de creación y validación del prototipo. La integración de sensores avanzados permitirá un monitoreo preciso del esfuerzo físico y la postura del usuario, asegurando la efectividad del dispositivo en situaciones reales.

\subsection{Factibilidad Económica}
El costo estimado para el desarrollo del prototipo es de aproximadamente \$10,000 MXN, significativamente menor que los exoesqueletos industriales comerciales que superan los \$50,000 MXN. Este presupuesto cubre materiales, componentes electrónicos, licencias de software necesarias y recursos para pruebas experimentales. La reducción de costos es posible gracias al uso de materiales accesibles y tecnologías de código abierto. Además, se prevé la posibilidad de obtener apoyo financiero a través de becas de investigación y colaboraciones con instituciones académicas y empresariales, lo que podría ampliar el presupuesto disponible para optimizar el prototipo y realizar pruebas más exhaustivas.

\subsection{Factibilidad Operativa}
El dispositivo está diseñado para ser fácil de usar, con una interfaz intuitiva que permite a los usuarios operar el exoesqueleto sin necesidad de formación extensa. Las pruebas con usuarios representativos confirmarán la facilidad de uso y la efectividad del dispositivo, asegurando que cumple con las necesidades de los trabajadores domésticos y profesionales. Además, la modularidad del diseño permite futuras mejoras y adaptaciones según los comentarios y resultados de las pruebas. Se establecerán protocolos de capacitación mínima para garantizar una adopción rápida y eficiente por parte de los usuarios finales, facilitando su integración en entornos laborales reales.

\subsection{Factibilidad Temporal}
El proyecto se planifica para completarse dentro de un plazo razonable, considerando las etapas de diseño, desarrollo, pruebas y optimización. La utilización de metodologías ágiles facilitará la gestión eficiente del tiempo y la adaptación a posibles desafíos durante el desarrollo. La división del proyecto en fases claras permite un seguimiento continuo del progreso y la implementación de ajustes necesarios. Se ha establecido un cronograma detallado con hitos específicos para cada etapa, garantizando el cumplimiento de los plazos y la asignación adecuada de recursos en cada fase del proyecto.
