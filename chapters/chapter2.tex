\chapter{Marco teórico y estado del arte}
% aqui van los antecedentes, teoría, modelos, métodos y metodología que emplearon los artículos y libros que se revisaron para el desarrollo del proyecto

\section{Marco teórico}
El marco teórico proporciona la base conceptual que respalda el desarrollo de este proyecto. La idea del exoesqueleto inspirado en una grúa requiere integrar conceptos de mecánica, ergonomía, biomecánica e inteligencia artificial, los cuales se detallan a continuación:

\subsection{Mecánica estructural}
La mecánica estructural estudia el comportamiento de estructuras sometidas a diferentes fuerzas. En este caso, se aplican principios de sistemas de grúas ligeras para garantizar que el exoesqueleto soporte las cargas propuestas sin comprometer su peso ni movilidad.

\subsection{Ergonomía y corrección postural}
La ergonomía analiza las interacciones entre los usuarios y su entorno, orientándose a reducir lesiones y mejorar el rendimiento. Este proyecto incorpora una faja ergonómica para promover una postura adecuada, mitigando el esfuerzo físico en la región lumbar.

\subsection{Inteligencia artificial y visión por computadora}
El uso de inteligencia artificial permite detectar herramientas en tiempo real mediante visión por computadora. Esto asegura que el exoesqueleto identifique, tome y manipule las herramientas sin intervención directa del usuario.

\subsection{Materiales y diseño ligero}
El diseño del dispositivo requiere materiales con una alta relación resistencia-peso, como aleaciones ligeras y polímeros reforzados, para garantizar que el exoesqueleto sea cómodo y funcional.

\section{Estado del arte}
El estado del arte describe los avances y aplicaciones existentes en exoesqueletos y tecnologías relacionadas, destacando las limitaciones que busca resolver este proyecto.

\subsection{Exoesqueletos en aplicaciones laborales}
Los exoesqueletos industriales han sido diseñados principalmente para soportar cargas pesadas en entornos de manufactura. Por ejemplo, el \textit{EksoVest} y el \textit{HeroWear Apex} se enfocan en reducir el esfuerzo en la parte superior del cuerpo, aunque no incluyen funciones de manipulación automatizada para herramientas ligeras.

\subsection{Sistemas de corrección postural}
Dispositivos como las fajas ergonómicas y exoesqueletos pasivos ayudan a mantener una postura correcta, pero carecen de integración con sistemas mecánicos activos o inteligencia artificial que permitan interacción autónoma con el entorno.

\subsection{Avances en inteligencia artificial aplicada a exoesqueletos}
La visión por computadora ha sido utilizada para tareas de detección de objetos en aplicaciones médicas e industriales. Tecnologías como \textit{YOLO} y \textit{OpenPose} han demostrado ser útiles para identificar movimientos y objetos en tiempo real, aunque su integración con sistemas ligeros y ergonómicos aún es limitada.

\subsection{Limitaciones de los diseños actuales}
Aunque los exoesqueletos actuales ofrecen soluciones robustas, suelen ser costosos, pesados y enfocados en cargas mayores a 10 kg. Esto deja sin atender a usuarios que necesitan dispositivos más accesibles para actividades con herramientas ligeras y tareas domésticas o laborales específicas.

