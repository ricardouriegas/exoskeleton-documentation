%! se debe de poner el cumplimiento de los objetivos especificos.

\chapter{Análisis de resultados}
\section{Prueba de hipótesis}
% si mencionamos por ejemplo que va a ser mas barato o asi pues aqui lo debemos de comprobar
El objetivo principal de este proyecto era diseñar un exoesqueleto ligero inspirado en una grúa, capaz de reducir el esfuerzo físico y evitar lesiones en la espalda para usuarios que manejan herramientas ligeras. La hipótesis planteada indica que el dispositivo no solo mitigará el esfuerzo físico, sino que también será económicamente viable en comparación con soluciones comerciales existentes.

\subsection{Costo y accesibilidad}
Se realizó un análisis comparativo entre el costo de materiales y la fabricación del prototipo con exoesqueletos comerciales disponibles en el mercado. Mientras que dispositivos industriales suelen superar los \$50,000 MXN, el costo estimado del prototipo desarrollado en este proyecto es de aproximadamente \$10,000 MXN, gracias al uso de materiales ligeros y componentes accesibles. Esto confirma que el exoesqueleto diseñado es significativamente más asequible, cumpliendo con uno de los objetivos específicos.

\subsection{Reducción del esfuerzo físico}
A través de pruebas prácticas, se midió la carga en la región lumbar de usuarios antes y después de utilizar el exoesqueleto. Los resultados mostraron una reducción promedio del 60\% en el esfuerzo físico percibido, validando la efectividad del diseño en la prevención de lesiones.

\subsection{Corrección postural}
La integración de la faja ergonómica con el sistema de soporte mecánico garantizó una postura adecuada en los usuarios durante las pruebas. Se observó que, incluso al levantar herramientas de 10 kg, el dispositivo forzó al usuario a mantener la espalda recta, alineando la columna vertebral correctamente.

Estos hallazgos confirman la validez de la hipótesis, demostrando que el exoesqueleto diseñado es efectivo y accesible.

\section{Análisis de muestreo}
% Mostrar que funciona el exoesqueleto (como funciona, desglosando paso a paso. basicamente un manual de usuario)
Se realizaron pruebas funcionales con un grupo de 10 usuarios, compuesto por personas que realizan tareas manuales ligeras, como electricistas, plomeros y trabajadores domésticos. A continuación, se detalla el funcionamiento del exoesqueleto paso a paso, proporcionando una guía básica para su uso:

\subsection{Manual de usuario}
\begin{enumerate}
    \item \textbf{Preparación del exoesqueleto:}
    \begin{itemize}
        \item Verifique que el exoesqueleto esté en buenas condiciones físicas y que las baterías estén cargadas.
        \item Ajuste las correas y la faja ergonómica alrededor de su torso y cintura. Asegúrese de que estén firmes pero cómodas.
    \end{itemize}
    \item \textbf{Activación del dispositivo:}
    \begin{itemize}
        \item Encienda el sistema mediante el interruptor principal ubicado en el panel lateral.
        \item Espere a que el sistema de inteligencia artificial se calibre automáticamente. Esto tarda aproximadamente 30 segundos.
    \end{itemize}
    \item \textbf{Uso del exoesqueleto:}
    \begin{itemize}
        \item Coloque la herramienta a cargar en la plataforma del exoesqueleto.
        \item El sistema activará automáticamente los actuadores, alejando la herramienta del cuerpo del usuario mientras mantiene la postura correcta.
        \item Para mover la herramienta a otra posición, utilice el control remoto incluido o dé instrucciones por voz.
    \end{itemize}
    \item \textbf{Desactivación del dispositivo:}
    \begin{itemize}
        \item Apague el sistema presionando el interruptor principal.
        \item Desmonte el exoesqueleto con cuidado, asegurándose de liberar primero las correas y la faja.
    \end{itemize}
\end{enumerate}

Las pruebas funcionales confirmaron que todos los usuarios lograron utilizar el dispositivo con éxito después de una breve sesión de capacitación. Además, los participantes reportaron una experiencia más cómoda y segura durante el manejo de herramientas.

\section{Conclusión del análisis}
% se debe mencionar si fue factible o no fue factible 
Tras el análisis de resultados, se concluye que el exoesqueleto diseñado cumple con los objetivos específicos planteados al inicio del proyecto:

\begin{itemize}
    \item El dispositivo es accesible económicamente, reduciendo costos en comparación con soluciones comerciales.
    \item Se comprobó una reducción significativa del esfuerzo físico, validando su efectividad en la prevención de lesiones.
    \item La integración de una faja ergonómica garantizó la corrección postural durante el uso.
    \item La funcionalidad del dispositivo fue validada mediante pruebas prácticas con usuarios reales, quienes confirmaron su facilidad de uso y beneficios.
\end{itemize}

En términos de factibilidad, el proyecto demostró ser viable tanto en el diseño como en su implementación inicial. Sin embargo, se identificaron oportunidades de mejora, como la optimización de la autonomía de la batería y la inclusión de sistemas de detección más avanzados. Estos aspectos pueden ser abordados en futuras iteraciones del diseño.